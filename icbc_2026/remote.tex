\section{Remote Mining on a Mobile Phone}\label{sec:remote_mining}

It is natural to move the two software parts in
Fig.~\ref{fig:local_mining} into two distinct machines, connected by a network.
This is because many users want to run a miner on their machine, and earn cryptos,
but only a few are typically interested in running a full blockchain peer on their machine.
This is shown in Fig.~\ref{fig:remote_mining}, where connections in code
are replaced by network connections.

\begin{figure}[b]
  \begin{center}
    \begin{tikzpicture}
      \draw [fill=VioletRed,thick] (0,0) rectangle +(3,1.5);
      \draw (1.5,0.75) node {{\scriptsize bockchain peer}};

      \draw [fill=lightred,thick] (5,0.35) rectangle +(2,0.8);
      \draw (6.0,0.75) node {{\scriptsize remote miner}};

      \draw [fill=verylightgray] (3.3,0.1) rectangle +(1.4,1.3);

      \draw[->, thick] (3,1.0) -- (5,1.0);
      \draw (4.0,1.2) node {{\scriptsize challenges}};

      \draw[<-, thick] (3,0.5) -- (5,0.5);
      \draw (4.0,0.25) node {{\scriptsize answers}};

      \draw (4.0,0.75) node {{\scriptsize\emph{network}}};

    \end{tikzpicture}
  \end{center}
  \caption{Peer and remote miner connected through a network.}\label{fig:remote_mining}
\end{figure}

However, this graphically simple modification might result useless or impractical,
depending on the specific characteristics of the mining algorithm and of the machine
where the remote miner must be run. Let us consider, in particular, the case when this
machine is a mobile phone.

\subsubsection*{Proof of work}

In this case, the challenge must include the full block that is being
created, since the nonce must be chosen in a way that maximizes the block's hash.
For Bitcoin, a block size can be up to 1~MB, or even 4~MB after the Segregated Witness upgrade.
This data must be exchanged at the creation of each block, which is hard in many
countries, where mobile internet is slow; moreover, most operators fix a maximum number of
GB exchange in the month, which would be easily exhausted; finally, the computational cost
of proof of work would drain the telephone's battery. These considerations are enough
to conclude that, at the moment, mining for proof of space, for a global blockchain like
Bitcoin, on a mobile phone, remains irrealistic.

\subsubsection*{Proof of stake}

In this case, mining is limited to the validators and is very cheap.
The exchanged data between peer and miner is very limited.
However, there is no reason for the validators to offload their very simple mining activity.
Therefore, mining on a mobile phone is theoretically possible for proof of stake, but it does
not seem to be of any practical interest.

\subsubsection*{Proof of space}

In this last case, mining requires the use of disk space, but not the use of significant
CPU power. At the time of this writing, mobile phones have an average disk space
of 256~GB. This is smaller than the average 1000~GB of desktop computers, but still significant.
Moreover, the proof of space of~\cite{Spoto25} uses very small challenges (around 64 bytes)
and answers (nonces, compacted into \emph{deadlines} of around 200 bytes~\cite{Spoto25}).
In particular, the block being created is not needed for mining. This makes this proof
of space algorithm an ideal choice for a remote miner running on a mobile phone:
no need of significant CPU power, enough disk space on the phone and little data exchanged
with the peer. This is why it has been the choice for the implementation of our
Mokaminter app.
