\section{Introduction}\label{sec:introduction}

Blockchain is a distributed network of computers (\emph{peers}).
Each peer receives transaction requests and eventually packs
such transactions into new \emph{blocks}. These blocks form a growing tree,
with each block pointing back to the previous one. Mining is the process of
block creation and includes a notion of chain quality that
determines which of the many tree branches eventually emerges
as the stable chain, on which all peers will agree,
therefore reaching \emph{consensus}. The mining
activity is normally remunerated, since the block creator has the right to
mint and earn a predetermined amount of cryptocurrency for each block created.

This very abstract description of a blockchain network is explicitly
generic wrt.\ two implementation choices:

\begin{enumerate}
  \item The exact notion of transaction. This is irrelevant in this paper.
    In general, a transaction is the specification of an
    operation that, once executed, modifies the state of an abstract machine.
    For instance, a transaction is a money transfer in Bitcoin, while in Ethereum
    (and in other blockchains that support Turing-complete smart contracts, up to gas
    exhaustion) it is the execution of some code over a shared state of objects.
  \item The chosen mining algorithm (also called consensus algorithm).
    Namely, Bitcoin pioneered the
    proof of work algorihtm, while more modern blockchains tend to use a
    proof of stake algorithm, and this paper focuses on a proof
    of space algorithm.
\end{enumerate}

In Bitcoin's proof of work, minted blocks must carry evidence of some computational
work that has been performed for their creation. The more work, the higher the
likelihood for the blocks to be integrated in the stable chain of the blockchain.
It is well-known how this led to high mining costs, in terms of electricity for CPU
computation and of data centers, and consequently to high ecological costs (resource
consumption and e-waste). Moreover, this led to the concentration of mining power in countries
where electricity is cheaper, against the original idea of full decentralization.

The proof of stake approach emerged as a reaction to this problem. Its idea is that
blocks are minted by a restricted set of peers only, called \emph{validators},
selected according to
some policy. This makes mining cheaper, efficient and ecologically sustainable.
Non-validators can stake on the success of the validators, but
it is generally accepted that proof of stake remains less decentralized than proof of work.
Nevertheless, it is the most used mining algorithm, at the moment.

Proof of space is a less-known algorithm. Its idea is that minted blocks must carry evidence
of the allocation of a large data structure on disk. The larger, the higher the likelihood
for the blocks to be integrated in the stable chain. Proof of space is decentralized like
Bitcoin's proof of work and ecologically sustainable like proof of stake, although the actual
degradation of disks (e-waste), due to mining, has not been studied up to now.

This paper tackles the problem of \emph{mobile mining}, that is,
mining on a mobile phone or tablet. In principle, mobile mining is appealing because:

\begin{itemize}
\item it leverages a huge existing base of devices, that otherwise remain idle
  for most of their time;
\item it makes mining available to a larger part of humanity, who owns a mobile
  phone but not a powerful connected computer;
\item it can be easily updated, thanks to the automatic update system of mobile apps;
\item it simplifies the start-up of new blockchains, since it is easier to convince people
  to install an app than it is to run and maintain a software on an always connected computer.
\end{itemize}

Nevertheless, mobile mining has not been used up to now because of an evident limitation
of mobile phones: their battery capacity and their network bandwidth are limited, in particular
in least favorite countries. As this paper will show, this limit is relative to the
specific mining algorithm. Proof of space emerges here as a niche solution that matches the
limitations of mobile mining.

This paper makes the following contributions:
%
\begin{enumerate}
\item it gives a theoretical presentation of mobile mining, its power and its limitations;
\item it shows how proof of space is much more adequate to mobile mining than other mining algorithms;
\item it presents Mokaminter, our actual Android app for mobile mining with proof of space;
\item it reports experimental results with Mokaminter (battery consumption, data exchanged).
\end{enumerate}

Mokaminter is an open-source app for Android, licensed under the Apache 2.0 terms. It can be installed
from Google Play~\cite{mokaminter-google-play}. Its source code is available from~\cite{mokaminter-source-code}.
