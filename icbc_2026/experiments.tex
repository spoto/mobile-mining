\section{Experiments}\label{sec:experiments}

Mokaminter has been installed on the personal phone of
a group of $11$ professors and students from the UFMA university in Brazil
(other potential users
owned only an iPhone and could not take part in the experiment).
Each of them created a miner for the Hotmoka blockchain
and let it run in the phone for a day. At the end, they collected
statistics about two quality metrics:
amount of data exchanged
and percentage of battery use, as reported by
the Android system in the settings page for Mokaminter.
The result is that Mokaminter exchanged only $0.808$~MB
per day and consumed only $0.27\%$ of battery power, on average.
These numbers can vary, of course, with the connection quality
and with the use of other apps in the phone, but the
experiment was meant to leave users free to operate on their phone as
they were used to, in order to measure costs in a realistic setting
that reflects real use patterns.

The quality of mobile internet in the area of Brazil where
the experiment has been conducted is low in comparison to other areas
of the country. Still, Mokaminter worked and mined without problems.
To push this test to te limit, Mokaminter has also been used during
an international flight, using the minimal free internet connection
on board. Even in this extremely adverse condition, Mokaminter
worked and mined.

In conclusion, these simple, yet limited experiments show that
Mokaminter is actually working, in various internet conditions,
and that its comsumption is minimal, in terms of bandwidth and
battery usage.
