\section{Related Work}\label{sec:related_work}

A simple search on Google Play lists quite a few Android apps for
\emph{crypto mining}. A deeper look makes it clear that such apps,
despite their descriptions, do not mine cryptocurrencies, but are either:

\begin{itemize}
  \item mobile dashboards
    to cloud pool mining applications.
    This is the case, for instance, of Ct Pool (\website{https://play.google.com/store/apps/details?id=net.ctpool}), ViaBTC (\website{https://play.google.com/store/apps/details?id=com.viabtc.pool}),
    ECOS (\website{https://play.google.com/store/apps/details?id=am.ecos.android}),
    Cloud mining Bitcoin Crypto (\website{https://play.google.com/store/apps/details?id=com.cmio.id1739008702352}),
    MineX (\website{https://play.google.com/store/apps/details?id=com.grenterprise.minexcryptomining})
    and many more;
  \item dubious games that promise Bitcoins in exchange of help for solving some tasks
    or in exchange of watching promotional content. This is the case, for instance,
    of Bitcoin Miner Earn Real Crypto (\website{https://play.google.com/store/apps/details?id=com.fumbgames.bitcoinminor}) and
    Crypto Fortune Tycoon: BTCMiner (\website{https://play.google.com/store/apps/details?id=com.minertycoon.ripple}).
\end{itemize}

This lack of a mobile mining app is not suprprising: mining for proof of work is not possible on a
mobile phone, that cannot compete with big hardware and cannot afford to drain the
battery for a heavy CPU-bound computation; mining for proof of stake is not possible
on a mobile phone since it is restricted to the validators and a validator's node and database
would never fit in a mobile phone; it remains mining for proof of space, which is what
this paper is going to present.
