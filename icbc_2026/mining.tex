\section{An Abstract Presentation of Mining}\label{sec:mining}

The different mining algorithms can be actually presented
inside a uniform setting. Namely, mining is a competitive game,
played, concurrently and independently, by each peer of a blockchain network.
Its goal is to choose which peer wins the right to create the \textbf{next}
block of the blockchain. This block must be \emph{valid}, that is, respect
all consensus rules, that are largely blockchain-specific. For example,
\textbf{next} must point back to the previous top of the blockchain; and
the transactions included in \textbf{next} must have valid signatures. But the
actual list of consensus rules is quite larger.

Mining algorithms can be presented in terms of:
%
\begin{itemize}
\item \emph{challenge}: given the current chain $\xi$,
  determine, deterministically,
  a challenge for \textbf{next};
\item \emph{answer}: create, if possible,
  a \textbf{next} that satisfies the challenge;
\item \emph{competition}: in case of more possible \textbf{next}, prefer the \emph{best}
  one \textit{wrt.\ }some notion of chain quality.
\end{itemize}
%
Each peer performs the mining algorithm, whispers the resulting \textbf{next} block to its
connected peers and receives the \textbf{next} block of its connected peers. In case
of more alternatives for \textbf{next}, the competition rule is used to choose the
emerging winner. This gives rise to a \emph{consensus algorithm}, which uses the mining
algorithm as a subroutine.

For proof of work, the general scheme for the mining algorithm becomes:
%
\begin{itemize}
\item \emph{challenge}: compute an integer $\difficulty$ from $\xi$
  and find a valid \textbf{next} such that
  $\hash(\mathbf{next})\ge\difficulty$;
\item \emph{answer}: create a \textbf{next} by rotating among possible
  values for its nonce field, until $\hash(\textbf{next})\ge\difficulty$;
\item in case of more possible \textbf{next}, prefer that with the largest hash.
\end{itemize}
%
The algorithm for computing $\difficulty$ is not relevant here.
Moreover, note that Bitcoin's proof of work is traditionally specified
in terms of the \emph{smallest} hash, but we prefer the equivalent
definition above, in terms of the largest hash,
to match the intuition that a higher $\difficulty$ makes mining harder.

For proof of stake, the generally accepted idea is that there is
\emph{no} mining~\cite{Kwon14}. However, a mining algorithm can be defined
also in this case:
%
\begin{itemize}
\item \emph{challenge}: compute an integer $i$ from $\xi$ (the index of the chosen
  validator) and find a valid \textbf{next};
\item \emph{answer}: if this algorithm is run by the $i$th validator,
  create a valid \textbf{next} block; otherwise, do not create any \textbf{next};
\item in case of more possible \textbf{next}, choose any of them, since
  the choice is irrelevant for what mining is concerned.
\end{itemize}
%
The algorithm for computing $i$ is irrelevat here: usually, a simple
round-robin is used. Moreover, what is reported above is the mining
algorithm, not the consensus algorithm, that is typically based on a Byzantine
protocol of pre-votes and pre-commits~\cite{Kwon14}.

Finally, proof of space, as presented in~\cite{Spoto25},
is based on the idea that \textbf{next} must contain
a data structure, called \emph{nonce}, that can be quickly recovered from disk from a
set of precomputed nonces (called, a \emph{plot file}). The mining algorithm in this case
becomes:
%
\begin{itemize}
\item \emph{challenge}: compute some data $\kappa$ from $\xi$;
\item \emph{answer}: create a valid \textbf{next} block, containing a nonce $\nu$
  from the plot file;
\item in case of more possible \textbf{next}, choose that with the smallest
  $\delay(\nu,\kappa)$, where $\nu$ is the nonce in \textbf{next}.
\end{itemize}
%
The algorithm for computing the challenge $\kappa$ is irrelevant here. In any case,
this $\kappa$ is used to choose a nonce from the plot file, whose quality measure,
the inverse of $\delay$, is maximized. The key point of this proof of space algorithm
is that the computation of the nonces (\emph{ie.}, of the plot file) is so expensive
that it cannot be done on-the-fly, at each mined block, but the plot file must be computed
instead once and for all, before starting mining, and kept on disk to answer, quickly, the mining
challenges. The consensus algorithm will use $\delay$ to wait for the time after which the block
becomes valid.

This abstract presentation of mining has the nice property of making it clear
that mining is a subroutine whose input is the challenge derived from the current
chain $\xi$ and whose output is the answer, that is, the information
needed to create \textbf{next}:
a nonce for proof of work and proof of space\footnote{Although the notion of nonce is
different in the two cases.}, and a Boolean for proof of stake
(create the block or rather let another validator create the block).
This is shown in Fig.~\ref{fig:local_mining}, where the peer that creates the node
and the miner that provides the answer to create te node are two distinct software
parts of the same tool, working on the same machine.

\begin{figure}[t]
  \begin{center}
    \begin{tikzpicture}
      \draw [fill=VioletRed,thick] (0,0) rectangle +(3,1.5);
      \draw (1.5,0.75) node {{\scriptsize bockchain peer}};

      \draw [fill=lightred,thick] (5,0.35) rectangle +(2,0.8);
      \draw (6.0,0.75) node {{\scriptsize local miner}};

      \draw[->, thick] (3,1.0) -- (5,1.0);
      \draw (4.0,1.2) node {{\scriptsize challenges}};

      \draw[<-, thick] (3,0.5) -- (5,0.5);
      \draw (4.0,0.25) node {{\scriptsize answers}};

    \end{tikzpicture}
  \end{center}
  \caption{Peer and local miner as a separate software parts in the same machine.}\label{fig:local_mining}
\end{figure}

