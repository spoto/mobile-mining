\section{Related Work}\label{sec:related_work}

Proof of space is a mining algorithm where disk (currently SSD) space
is used instead of CPU power.
It is energetically more efficient than proof of work and
more decentralized than proof of stake.
A preliminary
\emph{plotting phase}, when large data is allocated on disk,
through a potentially expensive process, is followed by a
\emph{mining phase}, when such data is used to quickly answer
block creation challenges received from blockchain peers:
the probability to create blocks for the stable chain
is proportional to the size of the allocated disk space.

Pioneering work~\cite{AtenieseBFG14,DziembowskiFKP15} introduced
formal proofs on persistent storage, showing that
CPU-bound on-demand recomputations (or \emph{replotting attacks}~\cite{BaigGP25}) are not viable,
due to high complexity pebbling graphs and Merkle trees:
any attempt to reduce the allocated space adds a prohibitive recomputation cost.
A subsequent evolution~\cite{RenD16}
uses \emph{stacked expanders} to reduce the implementation cost
and simplify proofs. The SpaceMint prototype~\cite{ParkKFGAP18},
now discontinued, showed the approach feasible and considered
mitigations to potential attacks, typical of
proof of space, such as mining on multiple chains or block/challenge grinding.
Chia~\cite{AbusalahACKPR17,CohenP19}, a more robust implementation, combines
space with verifiable delay functions, in a hybrid proof of space and time mechanism.
In Filecoin~\cite{Fil24}, parties commit space and show, repeatedly,
that they are still storing data in the committed space.
%While Chia is fully-permissionless, Filecoin is
%quasi-permissionless: anybody can participate, but under some requirements.
We build on the proof of space algorithm of Signum~\cite{signum},
previously Burstcoin, based on the precomputation of
plot files that later support quick block creation, as formalized
in~\cite{Spoto25}. Plot file are too expensive to recreate
on-the-fly for each block. We work with the generic blockchain
engine Mokamint~\cite{mokamint}, derived from Signum's algorithm, but generic \wrt{} the
notion of transactions, delegated to an application layer over Mokamint.

There is a lack of scientific research on the use of mobile phones for blockchain mining.
In~\cite{LeePSB20}, a blockchain system is proposed, to build local
industrial IoT blockchain networks, where drones are mobile miners
based on proof of work. This does not scale
to a global blockchain and to mobile phones, that
are not competitive against expensive specialized hardware and whose battery
would immediately get drained by the proof of work.
The same consideration applies to~\cite{SuankaewmaneeHN18}, that
presents a specialized blockchain for mobile commerce, with Android phones as
mobile miners based on proof of work.
Experiments are reported for blocks holding up to six transactions,
which is sensible for a local blockchain but irrealistic for a global one.
Despite its title, \cite{JiangLW22}~tackles a different problem, namely,
how to offload blockchain applications
to a cloud computing provider to make them usable
with the limited resources of the phone.
More generally, this approach allows edge devices, with strong hardware limitations,
to offload computationally expensive
tasks to external servers and to the cloud~\cite{SoundararajS25}.
The goal of the present paper is instead to perform mining \emph{in} the mobile phone
and offload the mining process from blockchain peers to many
mobile phones, expected to run a mobile mining app.

A simple search on Google Play reports a few Android apps for
\emph{crypto mining}. A deeper look makes it clear that,
despite their name, these apps are not miners, but
fall in either of the following categories.
(1) Mobile dashboards
    to cloud pool miners: mining occurs in the cloud and
    the app is just monitoring.
    This is the case, for instance, of Ct Pool~\cite{ctpool-google-play},
    ViaBTC~\cite{viabtc-google-play},
    ECOS~\cite{ecos-google-play},
    Cloud Mining Bitcoin Crypto~\cite{cloudminingbitcoincrypto-google-play}
    and MineX~\cite{minex-google-play}.
(2) Dubious games that promise Bitcoins in exchange of solving tasks
    or watching ads. This is the case, for instance,
    of Bitcoin Miner Earn Real Crypto~\cite{bitcoinminerearnrealcrypto-google-play}
    and Crypto Fortune Tycoon: BTCMiner~\cite{cryptofortunetycoonbtcminer-google-play}.
In particular, our research did not find any app for actually mining \emph{in} a phone.
