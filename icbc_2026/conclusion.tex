\section{Conclusion}\label{sec:conclusion}

This paper advocates the use of mining in a mobile phone, for blockchains based on
a proof of space mining algorithm. It describes our Mokaminter
mining app for Android and shows that it exchanges a negligeable
amount of data and consumes very little battery power,
which is important for an app that will likely run, continously, in
a phone. This shows that mobile mining in a phone is actually feasable.
Mokaminter is meant for blockchains based on the Mokamint engine~\cite{mokamint} and
the proof of space algorithm in~\cite{Spoto25}. Nevertheless, we think
that Mokaminter could be adapted to other blockchains based on
proof of space, such as Chia~\cite{AbusalahACKPR17},
as long as the amount of data exchanged by the protocol is limited.

It has been argued that proof of space might wear out disks and
possibly increase e-waste. It might also put pressure on the production
of SSDs and increase their price. This cannot be ruled out,
but it is difficult to assess it now, since proof of space is
still a niche technology.

If a proof of space blockchain reaches success, it is foreseeable that the size of the
disk space for mining would be so large that phones would be ruled out.
Nevertheless, a mobile mining app would remain a precious tool to start new blockchains
and lead them to the success that then makes such an app outdated.

