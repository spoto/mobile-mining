\section{Conclusion}\label{sec:conclusion}

This paper advocates the use of mining in a mobile phone,
for blockchains based on
a proof of space consensus algorithm. It describes our Mokaminter
mining app for Android and shows that it exchanges a negligeable
amount of data and consumes very little battery power,
which is important for an app that will likely run, continously, in
a phone. This shows that mobile mining in a phone is actually feasable.

Mokaminter is meant for blockchains based on the Mokamint engine~\cite{mokamint} and
the proof of space algorithm in~\cite{Spoto25}. Nevertheless, we think
that Mokaminter could be adapted to other blockchains as well, based
on other proof of space algorithms, such as Chia~\cite{AbusalahACKPR17},
as long as the amount of data exchanged by the protocol remains limited.

An iPhone version of Mokaminter might be available
in the future, but it is more complex to create than the current Android
version, since it is not possible
to leverage the existing Java libraries used for desktop miners.

It has been argued that proof of space might wear out disks and
possibly increase e-waste. It might also put pressure on the production
of SSDs and increase their price, in perspective. This cannot be ruled out,
but it is difficult to assess it now, since proof of space remains a niche
technology, at the moment.

In the lucky case of a proof of space blockchain reaching mainstream
adoptance and viral success, it is foreseeable that the amount of
disk space for mining would become so big that mobile phones could not
compete anymore against dedicated hardware. Despise of that, a mobile phone
mining app remains a precious tool to start up new blockchains
and increase their global mining power, until reaching the
success threshold that makes such an app outdated.

