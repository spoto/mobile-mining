\section{The Mokaminter App}\label{sec:mokaminter}

We have developed a mobile app, called Mokaminter,
for mining with proof of space.
This app creates and maintains a list of remote miners, each connected
to a blockchain peer (Fig.~\ref{fig:remote_mining}). It is possible
to create more miners for peers of the same or of different blockchains.
Mokaminter is an open-source app for Android, licensed under the Apache 2.0 terms. It can be installed
from Google Play~\cite{mokaminter-google-play}. Its source code is available from~\cite{mokaminter-source-code}.

Mokaminter is made for mining for peers of blockchains built through
the Mokamint generic blockchain engine~\cite{mokamint}, that implements
the proof of space algorithm in~\cite{Spoto25}. Mokamint is
a software tool that implements the networking and consensus layers of
a blockchain, while the application layer is left generic and can be
built on top of Mokamint. Currently, the Hotmoka software~\cite{hotmoka}
for running smart contracts in a subset of Java~\cite{Spoto19}
is available as an application
of Mokamint. This means that Mokaminter mines, at the moment,
for a blockchain with Java smart contracts whose consensus is based
on proof of space. This idea of a generic blockchain engine with
a pluggable application layer has been borrowed from Tendermint,
that pioneered the same approach.

The creation and activation of a miner with Mokaminter works as follows:
%
\begin{enumerate}
\item the user specifies the URI (universal resource identifier) of a
  blockchain peer and the size of the plot file for the miner
  (Fig.~\ref{fig:mokaminter_add_miner}, upmost part of the screen);
\item the app creates a new private/public key pair to identify the miner,
  represented, as usual from Bitcoin's time~\cite{Antonopoulos23},
  in terms of a $12$ words mnemonics (Fig.~\ref{fig:mokaminter_add_miner},
  lower part of the screen);
  \begin{itemize}
  \item alternatively, the user can insert the $12$ words of a key pair,
    if she wants to use an existing key;
  \end{itemize}
\item the app informs the user about the expected size of the plot file
  and asks for confirmation (Fig.~\ref{fig:mokaminter_confirmation});
\item if confirmed, the creation of the plot file runs in background;
\item once the plot file is created, the app adds a miner tile
  and activates a background process that mines by using the new plot file
  (Fig.~\ref{fig:mokaminter_mining}).
\end{enumerate}

\begin{figure}[t]
  \begin{center}
    \includegraphics[width=\textwidth/3]{pictures/mokaminter_add_miner}
  \end{center}
  \caption{Specification of the parameters for the creation of a miner.}
  \label{fig:mokaminter_add_miner}
\end{figure}

\begin{figure}[b]
  \begin{center}
    \includegraphics[width=\textwidth/3]{pictures/mokaminter_confirmation}
  \end{center}
  \caption{Confirmation is required before creating a new plot file.}
  \label{fig:mokaminter_confirmation}
\end{figure}

\begin{figure}[t]
  \begin{center}
    \includegraphics[width=\textwidth/3]{pictures/mokaminter_mining}
  \end{center}
  \caption{A new tile for a miner, working in the background.}
  \label{fig:mokaminter_mining}
\end{figure}

\subsubsection*{Security Considerations}
%
A lost or stolen mobile phone is a relatively frequent event.
If the private key of the miner were kept in the mobile phone, who finds
or stoles the phone would be able to access and move the cryptocurrency
accumulated while mining.
Because of this, it has been an explicit choice to avoid
storing the private key of the miners in the mobile phone.
Namely, after creating or importing the private/public key pair,
only the public key is used for creating the plot and the nonces,
and the app forgets the private key. On a negative side, this means
that the user must keep the $12$ words in a safe place and never lose them,
or otherwise access to the earned cryptocurrency becomes impossible.

\subsubsection*{Usability Considerations}
%
A mobile app has quite different usability considerations
from those of a desktop application.
%
A first consideration is about giving feedback to the user.
In a desktop application, the user could check warnings or error messages, as
well as the logs of the running application. In a mobile app this is not
comfortable. Therefore, Mokaminter gives feedback through
a heartbeat graphical effect that informs the user that the miner is
actually working, producing nonces. Moreover, network disconnection
is represented by turning the miner's tile grey in the list of miners.
%
Another usability issue is related to the insertion of the
$12$ words, if an existing key pair is imported.
Inserting words on a mobile phone software keyboard is extremely error-prone.
Therefore, Mokaminter suggests the possible words while typing, so that
the user can autocomplete them.
%
A last issue is about the specification of a plot file's size
that is too large to complete its creation in a reasonable time:
it would hang the mobile and drain its battery. To avoid this risk,
Mokaminter imposes a maximum size for the plot file, that can be
overridden, but only by an explicit choice of the user.

\subsubsection*{Technical Considerations}
%
Mokaminter connects to the blockchain peers by using websockets,
that perfectly match the continuous, bidirectional flow of challenges
and answers (Fig.~\ref{fig:remote_mining}). However, network connection in a
mobile phone is unreliable in comparison to a desktop computer, since
it gets lost frequently (thick walls, tunnels, network switches).
Mokaminter tries to reconnect whenever the connection is down or
no challenges have been received for an extended period of time. If it fails, it
tries again after a minute, to avoid draining the battery with
very frequent reconnection attempts.
%
Another technical issue is related to the background task for mining.
The Android system is becoming more and more restrictive with background
tasks, since they might consume battery power. There are official guidelines
for them, but these guidelines changed often
across different android versions. Moreover,
phone productors often add extra constraints. In practice, it is
frequent for Mokaminter to be turned off by the system overnight, when
the system enters a special \emph{doze mode}. If this happens, it is enough
to turn it on again in the morning. A better solution does not seem in sight,
at the moment.

The next section reports experiments with Mokaminter and measures
the main quality metrics for the app: the amount of data exchanged
and the percentage of CPU used for mining. If these were too high,
Mokaminter would not be usable.
