\section{The Mokaminter App}\label{sec:mokaminter}

We have developed the Mokaminter app for mining with proof of space.
It creates and maintains a list of remote miners, each connected
to a peer of the same or of different blockchains.
Mokaminter is an open-source app for Android, distributed
on Google Play~\cite{mokaminter-google-play} under
the Apache 2.0 license. Its source code is available from~\cite{mokaminter-source-code}.
An iPhone version might be available eventually,
but it is difficult to create since
it cannot leverage existing Java libraries from desktop miners.

Mokaminter works for blockchains built through
the Mokamint generic engine~\cite{mokamint}, that implements
the proof of space algorithm in~\cite{Spoto25}. Mokamint
implements the networking and consensus layers of
a blockchain, while the application layer is left generic and can be
built on top of Mokamint.
This idea of a pluggable application layer has been borrowed from Tendermint~\cite{Kwon14}.
Currently, the Hotmoka software~\cite{hotmoka}
for running smart contracts in a subset of Java~\cite{Spoto19}
is available as an application
of Mokamint. Therefore Mokaminter mines, at the moment,
for a blockchain with smart contracts in Java whose consensus is based
on proof of space, but it might mine for other blockchains in the future.

Creation and activation of a miner works as follows:
(1) the user specifies the URI (universal resource identifier) of a
  blockchain peer running Mokamint and the desired dimension of the plot file;
  %(Fig.~\ref{fig:mokaminter_add_miner}, above);
(2) the app creates a new private/public key pair to identify the miner,
  represented, as usual from Bitcoin's time~\cite{Antonopoulos23},
  in terms of a $12$ words mnemonics; % (Fig.~\ref{fig:mokaminter_add_miner}, below);
  alternatively, the user can insert the $12$ words of an existing key pair;
(3) the app informs the user about the expected size of the plot file
  and asks for confirmation; % (Fig.~\ref{fig:mokaminter_confirmation});
(4) if confirmed, the creation of the plot file starts in the background;
(5) once the plot file is created, the app adds a miner tile
  and activates a background process that mines by using the new plot file.
  %(Fig.~\ref{fig:mokaminter_mining}).

%\begin{figure}[t]
%  \begin{center}
%    \includegraphics[width=\textwidth/3]{pictures/mokaminter_add_miner}
%  \end{center}
%  \caption{Specification of the parameters for the creation of a miner.}
%  \label{fig:mokaminter_add_miner}
%\end{figure}

%\begin{figure}[b]
%  \begin{center}
%    \includegraphics[width=\textwidth/3]{pictures/mokaminter_confirmation}
%  \end{center}
%  \caption{Confirmation is required before creating a new plot file.}
%  \label{fig:mokaminter_confirmation}
%\end{figure}

%\begin{figure}[t]
%  \begin{center}
%    \includegraphics[width=\textwidth/3]{pictures/mokaminter_mining}
%  \end{center}
%  \caption{A new tile for a miner, working in the background. Eventually, its
%  cryptocurrency balance will increase.}
%  \label{fig:mokaminter_mining}
%\end{figure}

\subsubsection*{Security Considerations}
%
A lost or stolen phone is a relatively frequent event.
If the private key of the miner were kept in the phone, who finds
or steals the phone could also steal the cryptocurrency accumulated while mining.
Therefore, Mokaminter has been designed to avoid storing private keys in the phone.
Namely, after creating or importing the private/public key pair,
%(Fig.~\ref{fig:mokaminter_add_miner}),
only the public key is needed for creating the plot and for mining, and
the app forgets the private key. On the negative side, this means
that the user must not lose the $12$ words of the key pair,
or otherwise access to the earned cryptocurrency becomes impossible.

\subsubsection*{Usability Considerations}
%
A mobile app has quite different usability considerations
than a desktop application.
%
A first one is about feedback.
In a desktop application, the user could check warnings, error messages and logs.
In a mobile app, this is possible
but uncomfortable for most users. Therefore, Mokaminter gives feedback through
a heartbeat graphical effect that informs the user that it is
actually mining, producing nonces. Moreover, network disconnection
turns the miner's tile gray in the list of miners.
%
Another usability issue is related to the insertion of the
$12$ words, to import and existing key pair.
This is frustrating and error-prone in a mobile phone, through a software keyboard.
Thus, Mokaminter suggests the possible words while typing and the user can
autocomplete them.
%
A last issue is the risk of specifying a plot dimension that is
too large to complete the plot's creation in a reasonable time:
this would hang the phone and drain its battery. To avoid this risk,
Mokaminter imposes a maximum plot dimension, that can only be
overridden by an explicit choice of the user.

\subsubsection*{Technical Considerations}
%
Mokaminter uses websockets for network connections,
that perfectly match the continuous, bidirectional flow of challenges
and answers (Fig.~\ref{fig:remote_mining}). But network connection in a
mobile phone is less reliable than in a desktop computer, since
it gets frequently lost (thick walls, tunnels, network switches).
Mokaminter automatically tries to reconnect when the connection is lost or
no challenges have been received for an extended period of time. If it fails, it
tries again after a minute, to avoid draining the battery with
very frequent reconnection attempts.
%
Another technical issue is related to the background task for mining.
The Android system is extremely restrictive with background
tasks, since they might drain the battery. There are official guidelines
for them, but they changed across different Android versions and
phone manufacturers often add extra constraints. Many phones
turn Mokaminter off overnight, when
the Android system enters a special \emph{doze mode}. If this happens, it is enough
for the user
to turn it on again in the morning. A better solution does not seem in sight, at the moment.

