\section{Related Work}\label{sec:related_work}

\subsection{Proof of Space}

Proof of space is a mining algorithm where disk (currently SSD) space
is used for mining, instead of CPU power.
It is meant to be energetically more efficient than proof of work and
more decentralized than proof of stake.
The common idea of the many approaches is to distinguish a preliminary
\emph{plotting phase}, when large data is allocated on disk,
through a potentially expensive process, from a subsequent
\emph{mining phase}, when such data is used to quickly answer
the block creation challenges received from the blockchain peers.
The probability that a miner creates the next block of the stabke chain
is proportional to the disk space allocated during the plotting phase.

Pioneering works~\cite{AtenieseBFG14,DziembowskiFKP15} introduce
formal models of proofs based on persistent storage, showing that
CPU-bound on-demand recomputations (later called
\emph{replotting attacks}~\cite{BaigGP25}) are not viable,
due to the use of pebbling graphs of high complexity and of Merkle trees:
any attempt to reduce the allocated space entails a prohibitive increase
of the cost for recomputation. A subsequent evolution~\cite{RenD16}
uses \emph{stacked expanders} to reduce the implementational costs
and simplify the proofs. The SpaceMint prototype~\cite{ParkKFGAP18},
nowadays discontinued, showed the actual feasibility of the approach,
with considerations about mitigations of potential attacks that are typical of
proof of space, such as mining on multiple chains and block/challenge grinding.
A more robust implementation is Chia~\cite{AbusalahACKPR17,CohenP19}, that combines
proof of space with verifiable delay functions, in a hybrid
proof of space and time mechanism.
A last implementation is Filecoin~\cite{Fil24}, where parties must
commit space and show, repeatedly, that they are still storing data in the
committed space. While Chia is fully-permissionless, Filecoin is
quasi-permissionless: anybody can participate, but only under some requirements.

The present paper builds on the proof of space algorithm of the Signum blockchain~\cite{signum}.
Signum, previously, Burstcoin, is based on an
algorithm where plot files support quick creation of new blocks, but require
themselves an expensive algorithm for their creation, so that they cannot be created
on-the-fly for each block. The formalization of this blockchain was missing until
recently~\cite{Spoto25}. In particular, this article works with the generic blockchain
engine Mokamint~\cite{mokamint}, derived from the mining algorithm of Signum,
but generic \emph{w.r.t.} the
notion of transactions, that is left to an application layer that runs on top of Mokamint.

\subsection{Mobile Apps for Mining in a Phone}

There is a lack of scientific research on the use of mobile phones for blockchain mining.
In~\cite{LeePSB20}, a blockchain system is proposed, to build local blockchains
for industrial IoT networks, where mobile miners are drones that mine blocks
through proof of work. Their approach does not scale
to a global blockchain and to mobile phones, since
the latters would not be competitive against expensive specialized hardware
and the mining process would immediately drain their battery.
The same consideration applies to~\cite{SuankaewmaneeHN18}, that
presents a specialized blockchain for mobile commerce
whose miners are Android phones that perform the proof of work.
Experiments are reported for blocks holding up to six transactions,
which is sensible for a local blockchain but irrealistic for a global one.
Despite its title, a different problem is tackled
in~\cite{JiangLW22}, namely, how to offload blockchain applications
to a cloud computing provider so that such applications become
usable with the limited resources of a mobile phone.
More generally, this approach allows edge devices, with strong limitations on
CPU power, battery and disk space, to offload computationally expensive
tasks to external servers and to the cloud~\cite{SoundararajS25}.
But this is not the goal of the present paper,
which is instead to actually perform mining \emph{in} the mobile phone
and to offload the mining process from the blockchain peers to the many
mobile phones that are expected to run a mobile mining app.

A simple search on Google Play reports quite a few Android apps for
\emph{crypto mining}. A deeper look makes it clear that such apps,
despite their description, are not miners of cryptocurrencies, but
they rather fall in either of the following two categories:
%
\begin{itemize}
  \item mobile dashboards
    to cloud pool mining applications: mining occurs in the cloud,
    the app is just monitoring.
    This is the case, for instance, of Ct Pool (\website{https://play.google.com/store/apps/details?id=net.ctpool}), ViaBTC (\website{https://play.google.com/store/apps/details?id=com.viabtc.pool}),
    ECOS (\website{https://play.google.com/store/apps/details?id=am.ecos.android}),
    Cloud mining Bitcoin Crypto (\website{https://play.google.com/store/apps/details?id=com.cmio.id1739008702352}),
    MineX (\website{https://play.google.com/store/apps/details?id=com.grenterprise.minexcryptomining})
    and many others;
  \item dubious games that promise Bitcoins in exchange of help for solving some tasks
    or in exchange of watching promotional content. This is the case, for instance,
    of Bitcoin Miner Earn Real Crypto (\website{https://play.google.com/store/apps/details?id=com.fumbgames.bitcoinminor}) and
    Crypto Fortune Tycoon: BTCMiner (\website{https://play.google.com/store/apps/details?id=com.minertycoon.ripple}).
\end{itemize}
%
In particular, after our research,
we could not identify any existing mobile app for mining \emph{in} a phone.
