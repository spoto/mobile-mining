\section{Introduction}\label{sec:introduction}

Blockchain is a distributed network of computers (\emph{peers}).
Each peer receives transaction requests and eventually packs
such transactions into new \emph{blocks}. These blocks form a growing tree,
with each block pointing back to the previous one. Mining is the process of
block creation and includes a notion of chain quality that
determines which of the many tree branches eventually emerges
as the stable chain (hence reaching \emph{consensus}). The mining
activity is remunerated, since the block creator has the right to
mint and earn a predetermined amount of cryptocurrency for each block created.

This very abstract description of a blockchain network explicitly
leaves two specific implementation choices generic:

\begin{enumerate}
  \item The exact notion of transaction
    is irrelevant in this paper.
    In general, a transaction is the specification of an
    operation that, once executed, modifies the state of an abstract machine.
    For instance, a transaction is a coin transfer in Bitcoin, while it is
    a code execution over a shared state of objects in Ethereum and in
    other blockchains that support Turing-complete smart contracts.
  \item The mining algorithm is a parameter of the blockchain definition
    and can be replaced, in principle. Namely, Bitcoin pioneered the
    proof of work algorihtm, while more modern blockchains tend to use a
    proof of stake algorithm. This paper will concentrate on a proof
    of space algorithm instead.
\end{enumerate}
