\section{Introduction}\label{sec:introduction}

Blockchain is a distributed network of computers (\emph{peers}).
Each peer receives transaction requests, that eventually packs
into new \emph{blocks}. These form a growing tree,
with each block pointing back to the previous one. Mining is the process of
block creation and includes a notion of chain quality that
decides which of the many tree branches eventually emerges
as the stable chain, on which all peers agree,
therefore reaching \emph{consensus}. Mining
is normally remunerated: block creators may
mint and earn a predetermined amount of cryptocurrency. % for each block that it creates.
%as well as to earn a small fee on the transactions included in the blocks.

This very abstract description is explicitly generic \wrt{} two implementation choices:
%
\begin{enumerate}
  \item The notion of transaction, irrelevant in this paper.
    It is the specification of an
    operation whose execution modifies the state of an abstract machine.
    For instance, a money transfer
    in Bitcoin~\cite{Nakamoto08} or
    the execution of some code over shared objects
    in Ethereum~\cite{AntonopoulosW25}.
    %(and in other blockchains that support Turing-complete smart contracts,
    %up to gas exhaustion).
  \item The mining algorithm. Namely, Bitcoin pioneered the
    proof of work algorihtm~\cite{DworkN92}, while more modern blockchains tend to use a
    proof of stake algorithm~\cite{Kwon14}, and this paper focuses on a proof
    of space algorithm~\cite{Spoto25}.
\end{enumerate}

In Bitcoin's proof of work,
minted blocks carry evidence of the performance of some computational
work for their creation. The more this work, the higher the
likelihood for the blocks to be integrated in the stable chain of the blockchain.
It is well-known that this generates high electricity costs for CPU
computation and data centers, with a high ecological impact (resource
consumption, e-waste). Moreover, this induces the concentration of mining power in countries
where electricity is cheap, against the idea of full decentralization.

Proof of stake emerged as a reaction to this. Its idea is that
only a restricted set of peers (\emph{validators}) may mint blocks.
This makes mining cheaper, efficient and ecologically sustainable.
Non-validators may stake on the success of validators. That introduces some
decentralization, but
it is generally accepted that proof of stake is more centralized than proof of work.
This is the most used mining algorithm, at the moment.

The idea of proof of space~\cite{AbusalahACKPR17,AtenieseBFG14,CohenP19,DziembowskiFKP15,ParkKFGAP18,RenD16,Spoto25} is that blocks carry evidence
of the allocation of a large data structure on disk for their creation.
The larger that structure, the higher the likelihood that
blocks get included in the stable chain. Proof of space is decentralized like
Bitcoin's proof of work and ecological like proof of stake, although % the actual
disk degradation (e-waste), due to mining, has not been studied up to now.

This paper focuses on \emph{mobile mining}, that is,
mining on a mobile phone or tablet. In principle, this is appealing because:
%
\begin{itemize}
\item it leverages a huge existing hardware base of devices, that otherwise remain idle
  for most of the time;
\item it introduces mining to a large portion of humanity, who owns a
  phone but not a powerful connected computer;
\item it can be easily updated, thanks to the automatic update system of mobile phones;
\item it simplifies the start-up of new blockchains, since it is easier to convince people
  to install an app than to run and maintain a software on an always connected computer.
\end{itemize}
%
Mobile mining has not been used yet because of obvious limitations
of mobile phones \wrt{} battery capacity and network bandwidth, in particular
in least favorite countries.
%This paper shows that
%proof of space emerges as a niche solution here, whose mining
%is less sensitive to these limitations.

This paper makes the following contributions:
%
\begin{enumerate}
\item it presents mobile mining, its power and limitations;
\item it shows how proof of space emerges as a niche solution here, whose mining
  is less sensitive to these limitations;
\item it presents Mokaminter, our actual Android app for mobile mining with proof of space;
\item it reports experiments with Mokaminter, discussing its battery consumption
  and amount of exchanged data.
\end{enumerate}
%
Mokaminter runs only a miner in the phone, not a full blockchain
peer, which would be prohibitive for the size of the blockchain,
that of its state database, the amount of data exchanged and the computational
cost. The blockchain peer is an external computer, to which the phone connects.

The rest of this paper is organized as follows.
Sec.~\ref{sec:related_work} discusses related work.
Sec.~\ref{sec:mining} gives a uniform description of various mining algorithms.
Sec.~\ref{sec:remote_mining} considers if such algorithms can be implemented
on a mobile phone connected to a blockchain peer.
Sec.~\ref{sec:mokaminter} presents the Mokaminter mobile app for mining
with proof of space.
Sec.~\ref{sec:experiments} confirms its low battery cost
and its reduced bandwidth.
Sec.~\ref{sec:conclusion} concludes.
