\section{Remote Mining on a Mobile Phone}\label{sec:remote_mining}

It seems useful to move the two software parts in
Fig.~\ref{fig:local_mining} to distinct machines connected by a network,
as in Fig.~\ref{fig:remote_mining}.
Namely, many users want to run a miner and earn cryptos,
but only a few are interested in running a full blockchain peer on a dedicated machine,
which requires technical skills, database space and a very good internet connection.
%
\begin{figure}[b]
  \begin{center}
    \begin{tikzpicture}
      \draw [fill=VioletRed,thick] (0,0) rectangle +(3,1.5);
      \draw (1.5,0.75) node {{\scriptsize bockchain peer}};

      \draw [fill=lightred,thick] (5,0.35) rectangle +(2,0.8);
      \draw (6.0,0.75) node {{\scriptsize remote miner}};

      \draw [fill=verylightgray] (3.3,0.1) rectangle +(1.4,1.3);

      \draw[->, thick] (3,1.0) -- (5,1.0);
      \draw (4.0,1.2) node {{\scriptsize challenges}};

      \draw[<-, thick] (3,0.5) -- (5,0.5);
      \draw (4.0,0.25) node {{\scriptsize answers}};

      \draw (4.0,0.75) node {{\scriptsize\emph{network}}};

    \end{tikzpicture}
  \end{center}
  \caption{Peer and remote miner connected through a network.}\label{fig:remote_mining}
\end{figure}
%
But this graphically simple modification might result useless or impractical,
depending on the mining algorithm and on the machine
where the remote miner will run. Let us consider, in particular, the case when
the mining algorithm is one of those discussed
in Sec.~\ref{sec:introduction} and the miner is a mobile phone.

\subsubsection*{Proof of work}
%
In this case, the challenge includes the full block being
created, since its nonce must maximize its hash.
For Bitcoin, a block's size is up to 4~MB after the Segregated Witness upgrade.
This data must be exchanged at \emph{each} block creation, impossible in many
countries where mobile internet is slow; moreover, most mobile operators limit
the exchanged GB per month; finally, the computational cost
of proof of work would drain the phone's battery. Consequently,
at the moment, mining with proof of work is irrealistic on a mobile phone,
for a global blockchain such as Bitcoin.

\subsubsection*{Proof of stake}
%
In this case, mining is very cheap and limited to the validators. The exchanged data is minimal.
But validators have no reason to offload their very simple mining activity.
Therefore, mining on a mobile phone is possible for proof of stake, but it does
not seem of any practical interest.

\subsubsection*{Proof of space}
%
In this last case, mining requires the use of disk space, but no significant
CPU power. Currently, mobile phones have an average disk space
of 256~GB, smaller than the average 1000~GB of desktop computers but still significant.
Moreover, the proof of space of~\cite{Spoto25} uses very small challenges (around 64~bytes)
and answers (nonces, compacted into \emph{deadlines} of around 200~bytes).
The block being created is not needed for mining. This makes that
algorithm ideal for a remote miner in a mobile phone:
no need of significant CPU power, enough disk space in the phone and little data exchange.
This is why it has been the choice for the implementation of our Mokaminter app.
